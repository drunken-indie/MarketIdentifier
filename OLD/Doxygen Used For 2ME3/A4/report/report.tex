\documentclass[12pt]{article}

\usepackage{graphicx}
\usepackage{paralist}
\usepackage{amsfonts}

\oddsidemargin 0mm
\evensidemargin 0mm
\textwidth 160mm
\textheight 200mm
\renewcommand\baselinestretch{1.0}

\pagestyle {plain}
\pagenumbering{arabic}

\newcounter{stepnum}

\title{Assignment 4}
\author{Muyideen Jimoh, MacID: jimohma}

\begin {document}

\maketitle

This game of battleship is playable by two players only.  This battleship game has three modules in total; The Board Module, The Ships Module and The Gameplay Module. The ship module is designed as an abstract data type, therefore it is template module where both players have access to the ships available in this module. There are three ships in total present in this module namely; Titanic, Hokage and Kazekage. These ships each have a method in the ship module that returns where a particular ship is located on the board. The location of any particular ship is in the form of a list of tuples which has a size of 4 meaning that all three types of ship owned by a player has an exact size of 4. The board module is one both players have access to in order to mark the coordinates where they hit or miss a ship after a fire attack. The size of the board is a 10x10 which means that a player could make a maximum of 100 attacks in an attemp to sink all the ships of its opponent. Finally, the module for the actual game of battleship between the two players is called Gameplay. In this module is where different methods that describes the state of the game are defined. This module has the gameOver method which basically checks to see if all the ships have been sunk and it returns a boolean. In this module is the method which takes care of the actual game-play between player 1 and player 2. Lastly, when a coordinate of a ship is guessed correctly( i.e an attack by a player hit a coordinate where the enemy's ship is located), that coordinate(a tuple) is removed from the list of tuples which represents the ships location.


\newpage

\section* {Board Module}

\subsection*{Module}

Board

\subsection* {Uses}

N/A

\subsection* {Syntax}

\subsubsection* {Exported Constants}

MAX\_INDEX = 10 {\it // Dimension in the x and y-direction of the 10 X 10 sized board}\\
LEAST\_INDEX = 1 {\it //Row or Col index for the first coordinate on the game board}


\subsubsection* {Exported Access Programs}

\begin{tabular}{| l | l | l | l |}
\hline
\textbf{Routine name} & \textbf{In} & \textbf{Out} & \textbf{Exceptions}\\
\hline
init & List &  & ~\\
\hline
get\_state & List, real, real & String & INVALID\_SHIP\_POSITION\\
\hline
set\_state & List, real, real, String & String & INVALID\_SHIP\_POSITION\\
\hline

\end{tabular}

\subsection* {Semantics}

\subsubsection* {State Variables}

$grid$: sequence of List\\

\subsubsection* {State Invariant}
$| grid |  =  \mbox{MAX\_INDEX}$

\subsubsection* {Assumptions}

None

\subsubsection* {Access Routine Semantics}


\noindent init:
\begin{itemize}
\item transition: $grid := < >$
\item exception: none
\end{itemize}

\noindent get\_state($grid, row, col$):
\begin{itemize}
\item output: $out := grid[row][col]$
\item exception: $exc := (i \in [ 0..|grid| ]  \wedge  j  \in [ 0..|grid| ]  \wedge   \mbox{LEAST\_INDEX} > i > \mbox{MAX\_INDEX}   \vee   \mbox{LEAST\_INDEX} > j > \mbox{MAX\_INDEX} \Rightarrow \mathrm{INVALID\_SHIP\_POSITION})$
\end{itemize}

\noindent set\_state($grid, row, col, player$):
\begin{itemize}
\item transition: $grid := (grid[row][col]  = player)$
\item exception: $exc := (i \in [ 0..|grid| ]  \wedge  j  \in [ 0..|grid| ]  \wedge   \mbox{LEAST\_INDEX} > i > \mbox{MAX\_INDEX}   \vee   \mbox{LEAST\_INDEX} > j > \mbox{MAX\_INDEX} \Rightarrow \mathrm{INVALID\_SHIP\_POSITION})$
\end{itemize}


\newpage

\section* {Ship Module}

\subsection*{Template Module}

The\_ships

\subsection* {Uses}

Constants

\subsection* {Syntax}

\subsubsection* {Exported Types}

Ships = ?

\subsubsection* {Exported Constants}

MAX\_GRID = 10 {\it //Maximum coordinate index in the x and y-direction of the board}\\
MIN\_GRID = 1   {\it //Least coordinate index in the x and y-direction of the board}\\
SHIP\_SIZE = 4  {\it //Constant size for all the ships }\\

\subsubsection* {Exported Access Programs}

\begin{tabular}{| l | l | l | l |}
\hline
\textbf{Routine name} & \textbf{In} & \textbf{Out} & \textbf{Exceptions}\\
\hline
new Ships & List, List, List & Ships & ~\\
\hline
ship\_Titanic & ~ & List & INVALID\_SHIP\_POSITION, WRONG\_SHIP\_SIZE\\
\hline
ship\_Hokage & ~ & List & INVALID\_SHIP\_POSITION, WRONG\_SHIP\_SIZE\\
\hline
ship\_Kazekage & ~ & List & INVALID\_SHIP\_POSITION, WRONG\_SHIP\_SIZE\\
\hline
\end{tabular}

\subsection* {Semantics}

\subsubsection* {State Variables}

$Titanic$: sequence of Tuple\\
$Hokage$: sequence of Tuple\\
$Kazekage$: sequence of Tuple\\

\subsubsection* {State Invariant}

None

\subsubsection* {Assumptions}

None

\subsubsection* {Access Routine Semantics}

new Ships($ship1, ship2, ship3$):
\begin{itemize}
\item transition: $Titanic, Hokage, Kazekage:= ship1, ship2, ship3$
\item output: $out := \mathit{self}$
\item exception: none
\end{itemize}

\noindent ship\_Titanic:
\begin{itemize}
\item output: $out := Titanic$
\item exception: $exc := (|Titanic| \neq \mathrm{SHIP\_SIZE} \Rightarrow  \mathrm{WRONG\_SHIP\_SIZE} ~ | ~ i \in [0..|Titanic| - 1] \wedge  \mbox{MIN\_GRID} > Titanic[i][0] > \mbox{MAX\_GRID}  \vee  \mbox{MIN\_GRID} > Titanic[i][1] > \mbox{MAX\_GRID} \Rightarrow \mathrm{INVALID\_SHIP\_POSITION})$
\end{itemize}

\noindent ship\_Hokage:
\begin{itemize}
\item output: $out := Hokage$
\item exception: $exc := (|Hokage| \neq \mathrm{SHIP\_SIZE} \Rightarrow  \mathrm{WRONG\_SHIP\_SIZE} ~ | ~ i \in [0..|Hokage| - 1] \wedge  \mbox{MIN\_GRID} > Hokage[i][0] > \mbox{MAX\_GRID}  \vee  \mbox{MIN\_GRID} > Hokage[i][1] > \mbox{MAX\_GRID} \Rightarrow \mathrm{INVALID\_SHIP\_POSITION})$
\end{itemize}

\noindent ship\_Kazekage:
\begin{itemize}
\item output: $out := Kazekage$
\item exception: $exc := (|Kazekage| \neq \mathrm{SHIP\_SIZE} \Rightarrow  \mathrm{WRONG\_SHIP\_SIZE} ~ | ~ i \in [0..|Kazekage| - 1] \wedge  \mbox{MIN\_GRID} > Kazekage[i][0] > \mbox{MAX\_GRID}  \vee  \mbox{MIN\_GRID} > Kazekage[i][1] > \mbox{MAX\_GRID} \Rightarrow \mathrm{INVALID\_SHIP\_POSITION})$
\end{itemize}

\newpage

\section* {Game-Play Module}

\subsection* {Template Module}

Gameplay

\subsection* {Uses}

Board, The\_Ships

\subsection* {Syntax}

\subsubsection* {Exported Types}

GamePlay = ?

\subsubsection* {Exported Constants}

MAX\_FIRE = 100 {\it //Maximum attack to destroy all opponent's ship}\\
SHIP\_SIZE = 4  {\it //Constant size for all the ships }\\
MAX\_HIT = 12 {\it // Exact number of hits needed in order to sink all of enemy's ship and win the game}

\subsubsection* {Exported Access Programs}

\begin{tabular}{| l | l | l | l |}
\hline
\textbf{Routine name} & \textbf{In} & \textbf{Out} & \textbf{Exceptions}\\
\hline
new GamePlay & List, List, List, List & GamePlay & ~\\
\hline
gameOver& List, List, List & boolean & ~\\
\hline 
shipSunk& List & boolean & ~\\
\hline
winner\_mssg& String & String & ~\\
\hline
play& ~ & String & ~\\
\hline
\end{tabular}

\subsection* {Semantics}

\subsubsection* {State Variables}

$\mathit{p1\_ship}$: sequence of List {\it // Contains the three ship positions for player 1}\\
$\mathit{p2\_ship}$: sequence of List {\it // Contains the three ship positions for player 2}\\
$\mathit{p1\_fire}$: sequence of Tuple {\it // Contains the attacks fired by player 1}\\
$\mathit{p2\_fire}$: sequence of Tuple {\it // Contains the attacks fired by player 2}

\subsubsection* {State Invariant}

$|p1\_ship| , |p2\_ship| = \mbox{SHIP\_SIZE}$\\
\noindent
$|p1\_fire| , |p2\_fire| = \mbox{MAX\_FIRE}$

\subsubsection* {Assumptions}
The RegionT constructor is called for each abstract object before any other access routine is called for that
object.  The constructor can only be called once.

\subsubsection* {Access Routine Semantics}

\noindent new GamePlay($ship\_P1, ship\_P2, fire\_P1, fire\_P2$):
\begin{itemize}
\item transition: $\mathit{p1\_ship}, \mathit{p2\_ship}, \mathit{ p1\_fire},  \mathit{ p2\_fire} := ship\_P1, ship\_P2, fire\_P1, fire\_P2$
\item output: $out := \mathit{self}$
\item exception: none
\end{itemize}

\noindent gameOver($count$):
\begin{itemize}
\item output: $\mathit{out} := (count == \mbox{MAX\_HIT})$
\item exception: none
\end{itemize}

\noindent winner\_mssg($winner$):
\begin{itemize}
\item output: $\mathit{out} := "Message"$
\item exception: none
\end{itemize}

\noindent play:
\begin{itemize}
\item output: $\mathit{out} := ?$
\item exception: none
\end{itemize}

\end {document}

